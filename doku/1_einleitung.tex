\chapter{Einleitung}

Im Rahmen der Studienarbeit sollte eine App als Parkleitsystem für die neun Parkmöglichkeiten um den Altstadtring in Amberg entwickelt werden. Neben der Anzeige der Parkflächen auf einer Karte, ist es wichtig, auch detaillierte Informationen anzuzeigen. Dabei handelte es sich um Kosten pro Stunde, den Namen des Parkhauses oder Parkplatzes, die Anzahl der verfügbaren, belegten und gesamten Parkplätze, sowie ein aufsteigender, gleichbleibender oder fallender Trend. Des Weiteren muss dem Nutzer auch mitgeteilt werden, ob und wann das Parkhaus geöffnet ist und ob es überhaupt befahren werden kann.

Neben der Anzeige der Daten sollte es auch die Möglichkeit geben, eine Navigation zur Parkmöglichkeit zu starten, sobald der Nutzer sich in einem bestimmten Radius um die Örtlichkeit befindet. Wenn das sogenannte Geofence betreten wird, soll neben der Navigationsmöglichkeit als Toast auch noch Text als Sprachausgabe ausgegeben werden, um den Nutzer darauf hinzuweisen, welches Parkhaus am nächsten ist und befahren werden könnte. Damit der Nutzer nun nicht durch unpassende Sprachausgabe in unangenehme Situationen gebracht wird, sollte die Möglichkeit gegeben sein, den Ton auszuschalten. Außerdem sollen die Parkmöglichkeiten auch als Liste angezeigt und favorisiert werden können.

Um die aktuellen Parkhausdaten zu bekommen, soll eine API verwendet werden, welche die Daten bereitstellt und regelmäßig aktualisiert. Die Daten sind im XML-Format und unter folgendem Link zu finden: \url{https://parken.amberg.de/wp-content/uploads/pls/pls.xml}

Im Folgenden werden die Architektur und die Implementierung vorgestellt und näher beschrieben. Wie die zu entwickelnde Anwendung zu starten ist, wird im letzten Kapitel beschrieben.