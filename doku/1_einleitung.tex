\chapter{Einleitung}

Im Rahmen der Studienarbeit sollte eine App als Parkleitsystem für die neun Parkhäuser um den Altstadtring in Amberg entwickelt werden. Neben der Anzeige der Parkflächen auf einer Karte, war es wichtig, auch detaillierte Informationen anzuzeigen. Dabei handelte es sich um Kosten pro Stunde, den Namen des Parkhauses oder Parkplatzes, die Anzahl der verfügbaren, belegten oder gesamten Parkplätze, sowie ein aufsteigender, gleichbleibender oder fallender Trend. Des Weiteren muss dem Nutzer auch mitgeteilt werden, ob und wann das Parkhaus geöffnet ist und ob es überhaupt befahren werden kann.

Neben der Anzeige der Daten sollte es auch die Möglichkeit geben, eine Navigation zum Parkhaus zu starten, sobald man sich in einem bestimmten Radius um die Örtlichkeit befindet. Wenn man das sogenannte Geofence betritt, soll neben der Navigationsmöglichkeit Text als Sprachausgabe ausgegeben werden, um den Nutzer darauf hinzuweisen, welches Parkhaus am nächsten ist und befahren werden könnte. Diese Information soll auch mittels Toast auf dem Bildschirm angezeigt werden. Damit der Nutzer nun nicht durch unpassende Sprachausgabe in unangenehme Situationen gebracht wird, sollte die Möglichkeit gegeben sein, den Ton auszuschalten. Außerdem sollen die Parkmöglichkeiten auch als Liste angezeigt und favorisiert werden können.

Um die aktuellen Parkhausdaten zu bekommen, soll eine API verwendet werden, welche die Daten bereitstellt und regelmäßig aktualisiert. Die Daten sind im XML-Format und hier zu finden: \url{https://parken.amberg.de/wp-content/uploads/pls/pls.xml}

Im Folgenden Verlauf dieses Dokuments werden zunächst die verwendeten Technologien beschrieben. Anschließend werden die Vorüberlegungen zur Architektur dargelegt und die darauf folgende Implementierung vorgestellt. Im Anschluss daran findet sich eine Beschreibung, welche zeigt, wie das Projekt zu starten ist.