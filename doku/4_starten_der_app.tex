\chapter{Starten der App}
Damit das Projekt auf dem eigenen Rechner genutzt werden kann, müssen zunächst alle verwendeten Bibliotheken installiert werden. Dies geschieht ganz einfach mit der Ausführung der Befehle aus \autoref{lst:downloadDependencies} in der Kommandozeile.

\begin{lstlisting}[caption={Diese Befehle müssen in die Kommandozeile eingegeben und anschließend ausgeführt werden, damit das Projekt verwendet werden kann. So werden die notwendigen Bibliotheken installiert.},captionpos=b, language=Java, label=lst:downloadDependencies]
	cd path/to/project
	
	yarn
\end{lstlisting}

% TODO Das hier muss aber noch getestet werden
% TODO für geofences background location, aber das geht nicht in expo go

Die Anwendung funktioniert auf Android. Gearbeitet wird auf einem ,Pixel 5 API 33' im Android Emulator.

Um die App zu starten, muss zunächst Android Studio heruntergeladen und installiert werden. Nach dem Start von Android Studio muss über den Device Manager ein passender Emulator heruntergeladen werden. Anschließend muss der Emulator gestartet werden. Danach können die Befehle aus \autoref{lst:startAndroid} in der Kommandozeile ausgeführt werden.
\begin{lstlisting}[caption={Um die Anwendung mit mit Android zu starten, sollte Android Studio heruntergeladen und installiert sein. Anschließend muss der passende Emulator über den Device Manager hinzugefügt und gestartet werden. Danach werden diese Befehle in die Kommandozeile eingegeben und ausgeführt.},captionpos=b, language=Java, label=lst:startAndroid]
	cd path/to/project
	
	yarn android
\end{lstlisting}