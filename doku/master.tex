\documentclass[12pt,oneside]{report}
\usepackage{listings}
\usepackage[T1]{fontenc}		
\usepackage[utf8x]{inputenc}
\usepackage[ngerman]{babel}

\usepackage{parskip}			

\usepackage[a4paper,		
	    left=2.5cm,				
	    right=2.5cm,			
	    top=1.5cm,				
	    bottom=1.5cm,			
	    marginparsep=5mm,		
	    marginparwidth=10mm, 	
	    headheight=7mm,			
	    headsep=1.2cm,			
	    footskip=1.5cm,			
	    includeheadfoot]{geometry}

\usepackage{fancyhdr}						
\pagestyle{fancy}							
\fancyhf{}									
\setlength{\headwidth}{\textwidth}			
\fancyfoot[R]{\thepage} 					
\fancyfoot[L]{\leftmark}					
\fancyhead[R]{\IhreArbeit}					
\fancyhead[L]{\IhrVorname\ \IhrNachname}	
\renewcommand{\chaptermark}[1]{			
  \markboth{Kapitel \thechapter. #1}{}}
\renewcommand{\headrulewidth}{0.5pt}		
\renewcommand{\footrulewidth}{0.5pt}		
\fancypagestyle{plain}{					
  \fancyhf{}								
  \fancyfoot[C]{\thepage}					
  \fancyhead[R]{\IhreArbeit}				
  \fancyhead[L]{\IhrVorname\ \IhrNachname}	
}

\usepackage{amsmath}			
\usepackage{amssymb}
\usepackage[intlimits]{empheq}

\usepackage[sc]{mathpazo}		
\usepackage{pifont}				

\usepackage[format=hang,		
            font={footnotesize},
            labelfont={bf},
            margin=1cm,
            aboveskip=5pt,
            position=bottom]{caption}

\usepackage{graphicx}							
\usepackage[svgnames,table,hyperref]{xcolor} 
\usepackage{tikz}								
\usetikzlibrary{positioning,arrows,plotmarks} 

\usepackage{microtype,relsize}			
\newcommand*{\Sperren}[1]{\textls*[100]{#1}}


\newcommand*{\IhrVorname}{Annika}
\newcommand*{\IhrNachname}{Stadelmann}
\newcommand*{\IhrStudiengang}{Medieninformatik}
\newcommand*{\IhreArbeit}{Studienarbeit App-Programmierung WiSe 2022/23}
\newcommand*{\IhrTitelDE}{Entwicklung eines Parkleitsystems}
\newcommand*{\IhrBearbeitungszeitraumVON}{22. November 2022}
\newcommand*{\IhrBearbeitungszeitraumBIS}{17. Januar 2023}


\usepackage[bookmarks, raiselinks, pageanchor,
            hyperindex, colorlinks,
            citecolor=black, linkcolor=black,
            urlcolor=black, filecolor=black,
            menucolor=black]{hyperref}
\hypersetup{pdftitle={\IhrTitelDE},
            pdfauthor={\IhrVorname\ \IhrNachname},
            pdfsubject={\IhreArbeit}}

\lstset{numbers=left, numberstyle=\tiny, numbersep=5pt}
\lstset{language=Java}



\begin{document}
  \pagenumbering{roman}
  \begin{titlepage}					
    \thispagestyle{empty}
    \begin{center}
      \Large
      Ostbayerische Technische Hochschule Amberg-Weiden\\
      Fakultät Elektrotechnik, Medien und Informatik\\[1cm]
      Studiengang \IhrStudiengang\\[1cm]
      \textbf{\IhreArbeit}\\[1cm]
      von\\[1cm]
      \IhrVorname\ \Sperren{\textbf{\IhrNachname}}\\[1cm]
      \textbf{\IhrTitelDE}\\[1cm]
    \end{center}
    \vspace*{11cm}
    \begin{tabbing}
      \underbar{Bearbeitungszeitraum:}\qquad\= von\qquad\=\IhrBearbeitungszeitraumVON\\
                                        \> bis      \>\IhrBearbeitungszeitraumBIS
    \end{tabbing}
  \end{titlepage}
  
  \tableofcontents
  \newpage
  
  \pagenumbering{arabic}
  \chapter{Architektur der App}

% quelle des pin icons: https://www.pngfind.com/mpng/bRobT_map-pin-comments-map-pin-icon-png-transparent/
  
  \phantomsection
  \addcontentsline{toc}{chapter}{Literaturverzeichnis}
  \begin{thebibliography}{10}
    \bibitem{quelle1} Schreiberling, Tim: ,,Bestseller-Buch'', 1.~Auflage, S.~13ff, Renner-Verlag, Musterstadt, 2011.
  \end{thebibliography}
  \newpage
  
  \phantomsection
  \addcontentsline{toc}{chapter}{Abbildungsverzeichnis}
  \listoffigures
  \newpage
\end{document}    